
\title{Industry herding behaviour in the US stock market between 1926 and 2022: A focus on the role of political cycles}


\author { Xolani Sibande\footnote{Corresponding Author. Department of Economics, University of Pretoria, Pretoria, South Africa; Email: xolaniss@gmail.com. Economic Research Department, South African Reserve Bank, Pretoria, South Africa.} \,\, 
Vassilios Babalos\footnote{Department of Accounting and Finance, University of Peloponnese, Kalamata,  Greece; Email: v.babalos@uop.gr.} \,\,
Riza Demirer\footnote{Department of Economics and Finance, Southern Illinois University Edwardsville, Edwardsville, IL 62026-1102, USA; Email: rdemire@siue.edu} \,\,
Rangan Gupta\footnote{Department of Economics, University of Pretoria, Pretoria, South Africa; Email: rangan.gupta@up.ac.za.}
}
\date{\today}
\maketitle

\begin{abstract}

In this paper, we investigate the herding behaviour of investors in the US stock market using daily data from 1929 to 2022. We employ the cross-sectional absolute deviation (CSAD) measure of herding proposed by \cite{christie1995following} and \cite{chang2000examination}. Our results suggest that herding behaviour is present in the US stock market and that it is more pronounced during periods of high market volatility or crisis periods. This was evident is almost all the industry groups. Furthermore,  we find mixed evidence for intentional herding in industries. The results further show that herding is more prevalent in democratic administration due to high risk aversion. That, therefore, risk-aversion is a key factor in driving herding in some industries. Our findings have important implications for policy-makers and investors alike. 

\end{abstract}

\noindent\textbf{Keywords}: Herding Behaviour, Financial Crises, Market Efficiency
\\
\textbf{JEL Codes}: G01, G14, G15, G41
\newpage