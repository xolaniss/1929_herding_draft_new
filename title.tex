
\title{Presidential politics and investor behavior in the stock market: Evidence from a century of stock market data}


\author { Xolani Sibande\footnote{Corresponding Author. Department of Economics, University of Pretoria, Pretoria, South Africa; Email: xolaniss@gmail.com. Economic Research Department, South African Reserve Bank, Pretoria, South Africa.} \,\, 
Vassilios Babalos\footnote{Department of Accounting and Finance, University of Peloponnese, Kalamata,  Greece; Email: v.babalos@uop.gr.} \,\,
Riza Demirer\footnote{Department of Economics and Finance, Southern Illinois University Edwardsville, Edwardsville, IL 62026-1102, USA; Email: rdemire@siue.edu} \,\,
Rangan Gupta\footnote{Department of Economics, University of Pretoria, Pretoria, South Africa; Email: rangan.gupta@up.ac.za.}
}
\date{\today}
\maketitle

\begin{abstract}

This paper examines the role of political cycles on investor behavior in the U.S. stock market with a particular focus on herding behavior. Using long history of stock market data spanning from 1926 to 2022 and employing the return-based herding tests of \cite{christie1995following} and \cite{chang2000examination}, we first show that herding is more pronounced during periods of high market volatility or crisis periods across all industry groups examined. While herding is primarily driven by fundamental information regarding the stock market, we also find evidence of intentional herding, particularly during crisis periods, due to non-fundamental market factors. More importantly, we establish a link between herd behavior and political cycles. We show that herding is more prevalent during Democratic administrations, in line with the risk-based model of \cite{pastor2020political} that associates Democratic presidencies with high risk aversion and thus high risk premia in equities. Furthermore, we present evidence of spurious herding during periods of high approval of presidential economic policies, while intentional anti-herding is also present mostly for production-based industries during such periods, implying heterogeneous effects of partisan politics on industry herding. Our findings provide novel insight for the link between political dynamics and investor behavior with important implications for policy makers and investors. 

\end{abstract}

\noindent\textbf{Keywords}: Herding Behaviour, Financial Crises, Market Efficiency, Political Cycles
\\
\textbf{JEL Codes}: G01, G14, G15, G41
\newpage